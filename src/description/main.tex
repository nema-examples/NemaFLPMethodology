\documentclass[hidelinks,twoside,10pt,letterpaper]{../note}

\bibliographystyle{siam}
% \usepackage{../nema}
\usepackage{/Users/mopg/Documents/work/nema/code-extensions/nema-latex-package/src/nema}
\begin{document}

\title{Methodology for Solving {Facility Location Problems}}

\author{Max Opgenoord}
\date{\today}

\makeatletter
\let\newtitle\@title
\makeatother

\maketitle

\pagestyle{mainmatter}
\thispagestyle{empty}

{\bfseries\noindent
The problem of determining where our new distribution centers should be located is a version of the Facility Location Problem, a well-researched problem within the operations research community.
The problem is modeled as a mixed integer linear program
to the fundamental problem structure.
This paper describes the particular mathematical formulation used to solve such FLP problems within our organization.
The example shown in this paper highlights the effectiveness of the approach.
}

\section{Introduction}
Outdoor Logistics, Inc. is growing fast and our current distribution network is no longer adequate, see \NemaCite{a2fb11eefb13642f}.
The research science team has been tasked with mapping out a new strategy for our distribution network.
As part of this effort, we have developed an algorithm to solve facility location problems (FLPs).
This paper describes our approach in detail.

Facility location problems are a class of optimization problems that are widely encountered in various fields such as logistics, resource allocation, and urban planning \cite{farahani2009facility}.
These problems involve determining the optimal placement of facilities to serve a set of demand points, subject to various constraints and cost objectives.
The goal is to find a solution that minimizes the total cost, which may include factors such as transportation costs, capacity constraints, and site selection criteria.

Over the years, various mathematical models and algorithms have been proposed to solve facility location problems, including linear programming, integer programming, heuristics, and metaheuristics.
However, the complex nature of these problems often requires the use of multiple methods to achieve satisfactory solutions.
For example, mathematical programming models may provide optimal solutions but can be computationally intractable for large-scale problems, while heuristics can be fast but may not find the optimal solution.

Lucky for us, as a relatively new company, our logistics network is still fairly small and finding a globally-optimal solution is computationally tractable.
Our approach involves formulating the problem as a mixed integer linear program.

In the following sections, we describe our approach in detail and solve a small example problem to demonstrate the approach.

\section{Methodology}

The overall goal here is to minimize costs for our network expansion, while meeting customer demand over several years.
We have projections available for customer demand in our most important cities for the next few years.
The outcome of the optimization problem is a plan for what facilities to open in what years and what the capacity of those facilities should be.

Specifically, the costs to minimize are: (1) transportation costs between facility location ($j$) and customer hub ($k$) for each year ($i$), representated as $C_{t,ijk}$, (2) operating costs for each facility ($C_{o,ij}$), (3) start-up costs ($C_{u,ij}$) and wind-down costs ($C_{d,ij}$).
Mathematically, this is represented as:
\begin{subequations}\label{eq:flp_optimization}
\begin{align}
    \min_{A_{ij}, \ V_{ij}, K_{ijk}, C_{u,ij}, C_{d,ij}} \quad \sum_{ijk} C_{t,ijk} & + \sum_{ij} \left(C_{o,ij}  + C_{u,ij} + C_{d,ij}\right)  \hspace{-3in}&&  \\
    \textrm{subject to} \quad  \sum_j K_{ijk} &= D_{ik} \quad && \forall i,k \label{eq:opt:customer-demand}\\
    \sum_k K_{ijk} &= V_{ij} && \forall k \label{eq:opt:network-demand}\\
    V_{ij} &\leq A_{ij} V_{j,\max} && \forall i,j \label{eq:opt:max-capacity}\\
    C_{u,ij} &\geq f_{u,j} \left (A_{i^+j} - A_{ij}\right )  && \forall i,j \label{eq:opt:start-up}\\
    C_{d,ij} &\geq f_{d,j} \left (A_{ij} - A_{i^+j}\right )  && \forall i,j \label{eq:opt:wind-down}\\
    C_{t,ijk} &= K_{ijk} f_{t,j} S_{jk} \quad && \forall i,j,k  \label{eq:opt:transportation_cost}\\
    C_{o,ij} &= A_{ij} f_{o,j} \quad && \forall i,j \label{eq:opt:operating-cost}\\
    V_{ij} &\geq 0 \\
    K_{ijk} &\geq 0 \\
    C_{d,ij}, C_{u,ij} &\geq 0 \\
    A_{ij} &\in \left\{0,1\right\}
  \end{align}
\end{subequations}
where $A_{ij}$ indicates whether facility $j$ is active in year $j$, $V_{ij}$ is the volume served by facility $j$ in year $i$, $K_{ijk}$ is the volume served by facility $j$ for customer $k$ in year $j$.

The demand for all customer hubs ($D_{ij}$) needs to be served by our network, which is what the constraint in \cref{eq:opt:customer-demand} represents.
Our facilities should also be able to meet that volume moving in the network, \cref{eq:opt:network-demand}.
The volume in each facility has be 0 if the facility is not active, or below the facility's maximum capacity, \cref{eq:opt:max-capacity}.
Start-up and wind-down costs only need to be paid when the facility's status (active/inactive) changes between years, \cref{eq:opt:start-up,eq:opt:wind-down}.
The transportation costs depend on the volume moving along each leg of the network ($K_{ijk}$), the cost of transporting a single package per mile, $f_{t,j}$ (which varies geographically), and the distance between facility and customer hub ($S_{jk}$), \cref{eq:opt:transportation_cost}.
Finally, the operating cost of a facility are fixed per facility ($f_{o,j}$), but obviously depend on whether or not the facility is active, \cref{eq:opt:operating-cost}.


Note that in our analysis, net-present value (NPV) is taken into account.
This is not highlighted in \cref{eq:flp_optimization} to simplify the mathematical representation here.
However, the NPV terms are trivial to add to \cref{eq:flp_optimization}.

Finally, the start-up and wind-down costs in the first year are dependent on whether or not the facility is currently operating.
This is not shown in \cref{eq:flp_optimization} for ease of representation, but trivial to add.

\section{Example Problem}

As an example, we look at solving a small facility location problem with 9 customers and 5 facility locations, over a 5 year time span.
Information about the facilities  is listed in \cref{tab:facility}.
Note that the Seattle and Denver location are already in operation.

\begin{table}[h]
\caption{Facility information used in example.}\label{tab:facility}
\Nemapgfplotstabletypeset[
    col sep=comma,
    every head row/.style={before row=\toprule,after row=\midrule},
    every last row/.style={after row=\bottomrule},
    columns/name/.style={%
      column name=Location,
      column type={l},
      string type%
    },
    columns/max_yearly_capacity_k_packages/.style={%
      column name=Yearly capacity,
      preproc/expr={1000*##1},
      fixed, dec sep align%
    },
    columns/yearly_operating_cost_k_USD/.style={%
      column name=Yearly operating cost (k\$),
      fixed, dec sep align%
    },
    columns={name,max_yearly_capacity_k_packages,yearly_operating_cost_k_USD}
]{paper-example-facilities-info}
\end{table}

The customer demand information is shown in \cref{tab:customer_demand}.


\begin{table}[h]
  \caption{Customer demand assumed for example, the data between 2025 and 2027 is ommitted from this table.}\label{tab:customer_demand}
\Nemapgfplotstabletypeset[
    col sep=comma,
    every head row/.style={before row=\toprule,after row=\midrule},
    every last row/.style={after row=\bottomrule},
    columns/name/.style={%
      column name=Location,
      column type={l},
      string type%
    },
    columns/package_demand_k_packages_2024/.style={%
      column name=Demand 2024,
      preproc/expr={1000*##1},
      fixed, dec sep align%
    },
    columns/package_demand_k_packages_2028/.style={%
      column name=Demand 2028,
      preproc/expr={1000*##1},
      fixed, dec sep align%
    },
    columns={name,package_demand_k_packages_2024,package_demand_k_packages_2028}
]{paper-example-customer-demand}
\end{table}

The final results are shown in \cref{tab:facility_results}.
Of note here is that even though the Seattle and Denver locations have enough capacity to handle all 2024 orders, it is still more optimal to already open a Chicago facility to handle East Coast orders.

\begin{table}[h]
  \caption{Used capacity per facility for the years 2024 and 2028. For brevity, the results between 2025 and 2027 are ommitted from this table.}\label{tab:facility_results}
\Nemapgfplotstabletypeset[
    col sep=comma,
    every head row/.style={before row=\toprule,after row=\midrule},
    every last row/.style={after row=\bottomrule},
    columns/name/.style={%
      column name=Location,
      column type={l},
      string type%
    },
    columns/used_capacity_2024/.style={%
      column name=Used Capacity 2024,
      fixed, dec sep align%
    },
    columns/used_capacity_2028/.style={%
      column name=Used Capacity 2028,
      fixed, dec sep align%
    },
    columns={name,used_capacity_2024,used_capacity_2028}
]{paper-example-facility-capacity-result}
\end{table}

\section{Conclusion}

This paper has laid out the underlying methodology for solving facility location problems at Outdoor Logistics.
The approach uses mixed integer linear programs to solve globally optimal FLPs with a fast turn-around time.
We take a multi-year approach and the outcome is a strategic plan for where the build facilities and where to shut facilities down.
This approach will help Outdoor Logistics save substantially on building out a new logistics network while not impacting our customer promises.

\bibliography{bibliography.bib}

\end{document}